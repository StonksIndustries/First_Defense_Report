% Written by Alexandre Cölsch

Tout d'abord, l'imagination du scénario de \gameName étant un point central du début du projet, il a été commencé très tôt dans la période du projet, et terminé assez tôt également. Le jeu devait se diviser sur 3 grands actes résumant l'histoire principale composés d'entre 5 à 7 parties, mêlant un scénario semi-linéaire et un grand monde ouvert qu'est la terre d'Azerith, une terre où se déroule tous les évènements du jeu.
\\

L'ensemble de la trame principale de \gameName exploite la totalité de la carte du jeu, avec au total 21 zones qui font office de biomes différents. La progression du joueur dans l'histoire du jeu se fait via un système de quêtes principales à suivre en plus de quelques quêtes secondaires permettant de récupérer plus d'équipement et même d'avoir un impact sur la fin de l'histoire du jeu.
\\

Lors de l'acte 1, qui sera la seule terminée pour la fin du projet, et par conséquent la plus développée de toutes, le joueur commencera le jeu dans le petit village d'Emberwood qui sera la totalité de la première partie, et cette partie fera entièrement office de tutoriel, où le joueur pourra choisir sa classe de personnage entre 9 classes au total. Ensuite, le joueur découvrira au fur et à mesure de ce tutoriel les différentes mécaniques du jeu via des discussions avec certains personnages, comme le maniement des différentes armes (hache, épée…), les équipements par classe (armure lourde, légère…), les éléments (le feu, l'eau, le vent…), les consommables (les potions de régénération, les potions de défense…), le commerce en ville (forgerons, marchands…) ou encore plus simplement les mouvements du personnage.
\\

La fin de cette partie se termine par un combat où le joueur apprendra à gérer les mouvements de son personnage en combat, à utiliser ses compétences, ses pouvoirs élémentaires, ses armes ou encore ses consommables. Cette première partie, en plus de faire office de tutoriel, posera les bases du lore de \gameName.
La deuxième partie emmène directement le joueur dans le grand bain en le mettant en plein milieu de nulle part, dans un biome appelé la Champignome Forest, où le joueur devra mettre en œuvre ce qu'il a appris lors de la première partie à Emberwood. Il fera la connaissance des créatures du biome, aussi bien hostiles qu'amicales, dont notamment une espèce de créature importante dans la zone, appelée Smurfcat, un petit chat bleu marchant sur deux pattes avec un champignon en guise de chapeau. Les Smurfcats, contrairement à la plupart des créatures que rencontrera le joueur au fil de son avancée dans le jeu, sont amicales envers le joueur et un point central de la trame principale de Champignome Forest. Ces créatures introduiront la mécanique de quête secondaire, où le joueur peut choisir d'aider les Smurfcats pour avoir de l'équipement bonus et un impact sur la fin du jeu en lien avec ces créatures.
\\

Si l'on résume la trame principale de cette deuxième partie, le joueur a été téléporté à la Champignome Forest des suites de sa défaite (obligatoire dans le scénario) lors de son combat à Emberwood. Dès lors, le joueur à l'instruction de se diriger vers Gravity Crater, l'obligeant à passer par l'Ancient City, pour prendre ce qu'on appelle “un brouillard quantique calibré” ,qui est une formation paranormale d'une importance capitale dans le lore du jeu, pour retourner à Emberwood. (C'est également la source de sa téléportation entre Emberwood et la Champignome Forest.
\\

Pour expliquer les trois dernières parties de l'acte 1 sans s'enfoncer dans le lore (qui prendrait trop d'importance dans ce rapport), le joueur se verra donc revenir à Emberwood où il sera refusé d'entrer avant de partir pour Ruined Runes. Là-bas, le joueur rentrera dans le premier donjon du jeu nommé Rune Cave au nord de la zone, où il y affrontera également le premier boss de l'histoire principale, nommé A.F.I.T., de son vrai nom Algorithme Fabricateur à Intervalle Tertiaire, qui pour le décrire est une sorte de grosse machine datant d'un millénaire maîtrisant l'élément mécanique, qui est un des éléments du jeu, et face à ce boss le joueur apprendra les mécaniques des combats de boss (qu'il avait déjà un peu appris lors de la partie 1).
Des suites de sa victoire face à l'A.F.I.T., le joueur récupère un parchemin qui déterminera la suite de l'histoire, et ce dernier devra traverser la zone sombre de la Black Forest et la zone enneigée du White Volcano, qui ont tous les deux des quêtes secondaires, pour atteindre le King's Palace, autrement dit le château du roi d'Azerith pour ensuite lancer l'acte 2.
\\

L'acte 2 de la trame principale est l'acte qui met en valeur le monde ouvert d'Azerith, où le roi a demandé au joueur de récupérer des objets aux quatres coins de la carte avant de les ramener à King's Palace. C'est clairement l'acte qui laisse le plus de liberté au joueur dans son exploration de la carte, même si cela reste un minimum linéaire pour pas que le joueur se perde. Dans cet acte, le joueur explorera plus de zones, affrontera plusieurs boss, apprendra de nouvelles mécaniques et obtiendra un meilleur équipement au fur et à mesure de sa progression dans l'histoire.
\\

L'acte 3 est l'acte final de l'histoire du jeu, et il est également l'acte le plus linéaire des trois. La difficulté globale sera encore plus haute, et le joueur devra se mettre à l'épreuve et avoir le bon équipement au bon moment pour espérer terminer le jeu. L'acte 3 a deux fins différentes qui n'ont que peu d'importance sur le post-game, qui est entièrement multijoueur. Ces fins se baseront sur les quêtes secondaires, rendant le jeu plus ou moins difficile lors de la dernière partie de l'acte 3.
\\

Lors de sa progression dans le jeu, le joueur pourra en apprendre un peu plus sur le lore du jeu en suivant les quêtes secondaires, en discutant avec des personnages où en récupérant des notes un peu partout sur la carte, notes rangées dans un carnet que le joueur a automatiquement sur lui.
\\

En dernier lieu, lorsque le joueur termine une quête, aussi bien principale que secondaire, il obtient des récompenses sous forme de bonus d'expérience, d'équipement, de consommable ou d'objets spéciaux, la plupart utilisés dans d'autres quêtes. Dans le futur, étant l'une des parties les plus avancées du projet, il faudra juste détailler encore plus les quêtes et implémenter tout l'acte 1 dans le jeu.
\\

